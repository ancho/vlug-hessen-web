% Copyright (C) 2018 Stefan Kropp <stefan.kropp@vlug-hessen.de> 
%
%      Permission is granted to copy, distribute and/or modify this
%      document under the terms of the GNU Free Documentation License,
%      Version 1.3 or any later version published by the Free Software
%      Foundation; with no Invariant Sections, no Front-Cover Texts, and
%      no Back-Cover Texts.  A copy of the license is included in the
%      section entitled "GNU Free Documentation License".

\documentclass[a4paper,10pt,ngerman,titlepage,twoside]{scrreprt}
\usepackage[inner=4cm,outer=2.5cm,top=2cm,bottom=2cm,includeheadfoot]{geometry}
\usepackage[utf8]{inputenc}
\usepackage[T1]{fontenc}
\usepackage[german]{babel}
\usepackage{float}
\usepackage{scrdate}
\usepackage{scrtime}
\usepackage{listings}
\usepackage{graphicx}
\usepackage{makeidx}
\usepackage[ngerman]{varioref}
\usepackage{hyperref}
\usepackage{tabularx}
\usepackage{multirow}
\usepackage{multicol}
\usepackage{color}
\usepackage{shorttoc}
\usepackage{makeidx}
\usepackage{listing}
\usepackage{tikz}
\usepackage{fancyhdr}
\usepackage{babelbib}
\usepackage[nonumberlist, acronym, toc, section]{glossaries}
\bibliographystyle{alpha}
\makeindex
\makeglossaries

\begin{document}
%------------------------------------------------ 
% Vorspann
%------------------------------------------------ 
\setcounter{secnumdepth}{0}
\pagenumbering{Roman}

\titlehead{vLUG-Hessen - Willkommen} 
\subject{Virtuelle Linux User Group Hessen}
\title{Willkommen bei der vLUG-Hessen}
\author{Author}
\date{\today} 
\publishers{Virtuelle Linux User Group Hessen \\
	\url{http://www.vlug-hessen.de}
}

\maketitle
\tableofcontents
\newpage

% Abkürzungsverzeichnis / Akronym 
\newacronym{acronym:vlughessen}{vlughessen}{Virtuelle Linux User Group Hessen}
\newacronym{acronym:Acro2}{Acro2}{Acro 2}
\newacronym{acronym:Acro3}{Acro3}{Acro 3}

% Begriffserklärung / Glossar glossary 
\newglossaryentry{glossary:Test}{name={Glossar}, description={Beschreibung}}

% \gls{glos:AntwD}

\begin{abstract}
Willkommen bei der virtuellen Linux User Group Hessen.
\end{abstract}


\chapter{Vorwort}
\section{Willkommen und vielen Dank}
Herzlich Willkommen bei der \gls{acronym:vlughessen}.  Wir freuen uns auf eine schöne Zusammenarbeit und bedanken uns jetzt schon mal für
deine Mithilfe.
\section{Über uns}
Die \gls{acronym:vlughessen} nutzt Dienste im Internet um mit einander zu kommunizieren.
\begin{itemize}
	\item Mailingliste
	\item Git-Server 
	\item IRC
	\item gnusocial
\end{itemize}
Es ist unser Ziel \index{Ziel} die Linux User über lokale Grenzen hinweg zusammen zu bringen und zusammen an Projekte (Softwareentwicklung und Dokumentation) zu arbeiten.
%------------------------------------------------ 
% Hautpteil
%------------------------------------------------ 
\newcounter{anhangcounter}
\setcounter{anhangcounter}{\value{page}}
\setcounter{secnumdepth}{10}
\pagenumbering{arabic}
\chapter{Die ersten Schritte}
\section{Mailingliste}
Als erstes kannst du dich auf unserer allgemeinen Mailingliste eintragen. Hierzu schreibst du eine E-Mail an
vlughessen-request@lists.tuxfamily.org mit dem Betreff subscribe \index{Mailinglist}.
Bist du auf der Mailingliste eingetragen, macht es Sinn auf der Liste kurz Hallo zu sagen und sich vorzustellen.
\chapter{tuxfamily}
Wir nutzen einige Dienste auf tuxfamily.org. Aus diesem Grund ist es vielleicht sinnvoll sich hier einen Account einrichten zu lassen.
\url{https://www.tuxfamily.org/en/subscribe}.
\section{Git}
\begin{lstlisting}[caption={Git repos}, label={listing:gitreposl}]
git clone ssh://<user>@git.tuxfamily.org/gitroot/vlughessen/shellscripts.git
git clone ssh://<user>@git.tuxfamily.org/gitroot/vlughessen/dokumentation.git
\end{lstlisting}
\chapter{Kommunikation}
\section{Mailingliste}
Wichtige Informationen sollten immer über die Mailingliste kommuniziert werden: vlughessen@lists.tuxfamily.org
\section{IRC}
Wir haben auf Freenode ein IRC-Chat-Raum: \#vlughessen. Hier kann jeder kommen wann und wie er möchte.
Wir versuchen euch einen Stammtisch hinzubekommen.
Unser Stammtisch ist jeden 1. Dienstag im Monat im IRC Freenode. Treffen ist ab 19:00 Uhr in \#vlughessen.
\section{GNUSocial}
Auf GNUSocial findest du uns auf \url{https://gnusocial.de/group/vlughessen}
\section{E-Mail}
Du hast Fragen? Dann schreib einfach eine E-Mail an die Mailingliste oder direkt an uns kontakt@vlug-hessen.de.
% Beginn Anhang
\appendix
\chapter{Anhang}
\section{Verzeichnisse}
\listoffigures
\lstlistoflistings
\listoftables
%\printbibliography
\printglossaries
\printindex
\bibliography{Literatur}
\end{document}

